\documentclass{article}

\usepackage{amsmath, amsthm, amssymb, amsfonts}
\usepackage{thmtools}
\usepackage{graphicx}
\usepackage{setspace}
\usepackage{geometry}
\usepackage{float}
\usepackage{hyperref}
\usepackage[utf8]{inputenc}
\usepackage[english]{babel}
\usepackage{framed}
\usepackage[dvipsnames]{xcolor}
\usepackage{tcolorbox}

\colorlet{LightGray}{White!90!Periwinkle}
\colorlet{LightOrange}{Orange!15}
\colorlet{LightGreen}{Green!15}

\newcommand{\HRule}[1]{\rule{\linewidth}{#1}}

\declaretheoremstyle[name=Definicion,]{thmsty}
\declaretheorem[style=thmsty,numberwithin=section]{theorem}
\tcolorboxenvironment{theorem}{colback=LightGray}

\declaretheoremstyle[name=Proposition,]{prosty}
\declaretheorem[style=prosty,numberlike=theorem]{proposition}
\tcolorboxenvironment{proposition}{colback=LightOrange}

\declaretheoremstyle[name=Principle,]{prcpsty}
\declaretheorem[style=prcpsty,numberlike=theorem]{principle}
\tcolorboxenvironment{principle}{colback=LightGreen}

\setstretch{1.2}
\geometry{
    textheight=9in,
    textwidth=5.5in,
    top=1in,
    headheight=12pt,
    headsep=25pt,
    footskip=30pt
}

% ------------------------------------------------------------------------------

\begin{document}

% ------------------------------------------------------------------------------
% Cover Page and ToC
% ------------------------------------------------------------------------------

\title{ \normalsize \textsc{}
		\\ [2.0cm]
		\HRule{1.5pt} \\
		\LARGE \textbf{\uppercase{Apuntes Gestión de Proyectos}
		\HRule{2.0pt} \\ [0.6cm] \LARGE{} \vspace*{10\baselineskip}}
		}
\date{}
\author{\textbf{Author} \\ 
		B.L.B}

\maketitle
\newpage

\tableofcontents
\newpage

% ------------------------------------------------------------------------------
\section{Conceptos básicos}
\subsection{Proyecto}
\begin{theorem}
Esfuerzo temporal que se lleva a cabo para crear un producto, servicio o resultado único.
\end{theorem}
Características:
\begin{description}
\item[Temporal]
	\begin{itemize}
		\item Siempre tiene un inicio y un final.
		\item Inicio: se decide \textbf{qué} hacer.
		\item Final: se obtiene el producto/servicio.
		\item Algunos se cancelan
	\end{itemize}
\item[Crean un producto único y medible]
\item[Progresivo]
\end{description}
Un proyecto \textbf{no} son siempre críticos o estratégicos, \textbf{no} son procesos y \textbf{no} son siempre exitosos.

\subsection{Dirección de Proyectos}
\begin{theorem}
Aplicación de conocimientos, habilidades, herramientas y técnicas a las actividades de un proyecto para satisfacer sus requisitos.
\end{theorem}
El \textit{director del proyecto} es la persona responsable de alcanzar los objetivos del proyecto. Para ello necesita:
\begin{itemize}
	\item Conocimientos.
	\item Capacidad para sacar adelante el trabajo.
	\item Habilidades interpersonales.
\end{itemize}

\subsection{Interesados (Stakeholders)}
\begin{theorem}
Personas y organizaciones implicadas de forma activa en dicho proyecto, y aquellos cuyos intereses pueden verse afectados de manera positiva o negativa como resultado del desarrollo del proyecto. 
\end{theorem}
Por ejemplo: equipo de proyecto, empleador, usuarios, sociedad, etc.

\subsection{Conceptos básicos}
\begin{theorem}
\textbf{Requisitos:} Necesidades y expectativas de los interesados.
\end{theorem}

\begin{theorem}
\textbf{Objetivo: } Qué queremos obtener como consecuencia del desarrollo del proyecto.
\end{theorem}

\begin{theorem}
\textbf{Alcance: }Acciones llevadas a cabo para conseguir los objetivos.
\end{theorem}

\begin{theorem}
\textbf{Recursos: } Elementos necesarios para el desarrollo del proyecto. Estos recursos son limitados y hay que gestionarlos de manera adecuada.
\end{theorem}

\begin{theorem}
\textbf{Calidad: }Conjunto de características que deseamos que cumplan los requisitos de los productos.
\end{theorem}

\begin{theorem}
\textbf{Riesgo: }Todo aquello que puede ir mal en la consecución de los objetivos.
\end{theorem}

\subsection{Restricciones del proyecto}
En todo proyecto se sufrirá el efecto de seis tipos de restricciones: tiempo, coste, alcance, recursos, calidad y riesgo. Cada vez que ocurra un cambio en el proyecto, habrá que mirar cuál es el efecto que tiene sobre cada uno de los tipos de restricciones.

\subsection{Proceso:}
\begin{theorem}
Serie de acciones que dan lugar a un resultado.
\end{theorem}
Un proyecto se compone de procesos relacionados entre sí mediante sus entradas y salidas. Toda tarea de un proyecto pertenece a un proceso.
Hay dos categorías principales:
\subsubsection{Procesos orientados a producto}
Contemplan la especificación, diseño y creación del objetivo del proyecto y varían según el área de aplicación y el objetivo. Del objetivo se decide un ciclo de vida y este especifica los procesos a ejecutar.
\subsubsection{Procesos de Planificación y Gestión}
Tratan de la planificación, organización y gestión del proyecto. La mayoría de estos procesos son aplicables a todos los proyectos. La guía del PMBOK define 47 procesos y los clasifica de la siguiente manera:
\begin{itemize}
	\item 5 grupos de procesos (en base a su funcionalidad).
	\begin{itemize}
		\item Procesos de inicio: Se utilizan para \textbf{definir} un nuevo proyecto. Se especifican decisiones de alto nivel (escoger director, justificación del proyecto, restricciones, supuestos...). Se identifican a los \textbf{interesados}
		\item Procesos de planificación: tienen como objetivo obtener el plan de proyecto (identificar tareas, estimación de esfuerzo, planificar recursos...).
		\begin{itemize}
		\item \textbf{Plan de proyecto:} documento coherente y firme que sirve de guía para la ejecución y control del proyecto.
		\item \textbf{DOP:} se desarrolla en las primeras fases del proyecto, contiene menos información que el plan de proyecto y debe ser aprobado formalemente. Incluye descripción, objetivos, alcance, planificación temporal, riestos y método de trabajo.
		\end{itemize}
		\item Procesos de ejecución: desarrollan el trabajo necesario para conseguir los objetivos establecidos siguiendo el plan del proyecto.
		\item Procesos de seguimiento y control: se encargan del seguimiento del trabajo realizado. Se compara el estado de las tareas, recursos utilizados, etc con el plan del proyecto. Cuando sea necesario, se solicitarán modificaciones. 
		\item Procesos de cierre: se verifica que han finalizado los procesos anteriores, se obtiene el visto bueno del cliente...
	\end{itemize}
	\item 10 áreas de conocimiento (en base a su temática).
	\begin{itemize}
		\item Integración
		\item Alcance
		\item Tiempo
		\item Coste
		\item Calidad
		\item Recursos Humanos
		\item Comunicación
		\item Riesgos
		\item Adquisiciones/Suministros
		\item Interesados
	\end{itemize}
\end{itemize}
\section{Reuniones}
En la primera reunión interna se hará partícipe al equipo de los objetivos de proyecto, se asignarán responsabilidades, se organizará la gestión del proyecto y se preparará la reunión con el cliente.
\subsection{Reuniones con cliente}
Se revisarán objetivos, se concretará qué hacer y qué no hacer y se concretarán los requisitos. 
Se intentará conocer al cliente antes de la reunión, y cuando se trate de aspectos técnicos, debemos intentar dirigir la reunión nosotros.

\subsection{Reuniones de desarrollo}
Sirven para especificar el trabajo a desarrollar y distribuir responsabilidades. Los posibles asistentes son: los miembros del equipo, el cliente, terceros implicados...

\subsection{Reuniones de seguimiento}
Son sólo para miembros del equipo. Se dispondrá de la programación del proyecto y de la asignación de recursos en cada instante. Se conocerán los costes hasta el momento. 
Es necesario realizar un seguimiento de lo planificado, tomando las medidas oportunas cuando:
\begin{itemize}
	\item Se produzcan retrasos.
	\item Costes por encima de lo planificado.
	\item Se contravenga algunas condiciones acordadas que fueron base en la decisión de realizar este proyecto. 
\end{itemize}
Tan pronto como se observen desviaciones habrá que replanificar o renegociar el plan del proyecto con los clientes. 
\subsection{Roles}
\begin{itemize}
	\item Coordinador: planifica la reunión, define la agenda, el orden del día, controla el desarrollo de la reunión y la evalúa. 
	\item Secretario: convoca la reunión, prepara los materiales y redacta el acta de reunión.
	\item Asistentes
\end{itemize}
\subsection{Antes de la reunión}
Planificar la reunión:
\begin{itemize}
	\item Objetivo: ¿Para qué hacemos la reunión?
	\item Agenda: ¿Qué temas vamos a tratar? ¿Qué queremos transmitir?
	\item Asistentes: ¿A quién va dirigida la reunión? ¿Quién pondrá pegas durante la reunión?
	\item Materiales: ¿Qué documentación hay que preparar?
	\item Organización de la agenda: orden de cada tema, tiempo que dedicaremos a cada uno.
	\item Lugar, fecha, hora, duración.
\end{itemize}
\section{Objetivos y alcance (06/02/2024)}

Son elementos clave para definir y planificar el trabajo:
\begin{itemize}
	\item Objetivos (QUÉ): resultados esperados. ¿Qué queremos conseguir?
	\item Alcance (CÓMO): ámbito de acción y requisitos que debe cumplir. ¿Qué hay que hacer para desarrollar el proyecto?
\end{itemize}

Deben estar alineados entre sí y con la misión del proyecto, y coherentes con los recursos y restricciones existentes. 

\subsection{Definir objetivos}
Pasos:
\begin{itemize}
	\item Identificar interesados: stakeholders
	\item Identificar qué quiere cada interesado
	\item Tener en cuenta solo los intereses convenientes para el proyecto. 
\end{itemize}
\subsection{Redactar objetivos}
La redacción de los objetivos es única para todos los interesados.
\begin{itemize}
	\item Objetivos SMART: \textbf{específifos}, \textbf{medibles}, \textbf{alcanzables}, \textbf{relevantes}, \textbf{temporales} (se utilizan verbos para redactarlos)
	\item Objetivos \textbf{generales} y \textbf{específicos}:
	\begin{itemize}
		\item Generales: declaraciones claras y concisas de los propósitos del proyecto.
		\item Específicos: se definen las metas a corto plazo que deben cumplirse para lograr el objetivo general de forma más detallada.
	\end{itemize}
	\item Errores típicos:
	\begin{itemize}
		\item Dividir los objetivos por interesados.
		\item Confundir tareas con objetivos.
		\item Definir objetivos poco realistas o inalcanzables.
	\end{itemize}
\end{itemize}

\subsection{Alcance}

Una vez definidos los objetivos, hay que definir qué pasos dar para cumplirlos.
\begin{itemize}
	\item El alcance debe ser claro, completo y consensuado con las partes interesadas
	\item Alcance del producto: características y funcionalidades del producto final: Documento de requisitos.
	\item Alcance del proyecto: tareas que se llevarán a cabo para conseguir el producto: Estructura de Descomposición del Trabajo (EDT).
\end{itemize}

\subsection{Redactar alcance}
\begin{itemize}
	\item Ver qué tareas son necesarias para conseguir los objetivos
	\item Dividir las tareas en subtareas más fáciles de estimar y asignar al personal.
	\item Las tareas se incluyen en una estructura jerárquica llamada \textbf{EDT}.
\end{itemize}

\subsection{EDT}
\begin{itemize}
	\item Nivel 1: Proyecto
	\item Nivel 2: Fases, Paquetes de trabajo: INICIO, DISEÑO, IMPLEMENTACIÓN, PRUEBAS, GESTIÓN
	\item Nivel 3: Paquetes de trabajo, tareas que cuelgan del nivel 2.
	\item Nivel 4: Paquetes de trabajo, subtareas que cuelgan del nivel 3.
\end{itemize}
Todas las tareas del EDT forman el alcance del proyecto, y lo que no está en el EDT no es parte del proyecto (todo debe estar definido aquí). 
Gestionar el proyecto (el seguimiento y supervisión) debe estar incluído, al fin y al cabo es tiempo invertido en el proyecto. 

Se pueden establecer una serie de entregas intermedias (entregables): para el cliente (presupuestos, prototipos...) o para el equipo de trabajo (DOP, análisis...) que ayudan a establecer hitos o fechas clave. 

Entregables:
\begin{itemize}
	\item DOP: una vez finalizada la tarea "Elaboración del DOP".
	\item Presupuesto: una vez finalizada la tarea "Organización".
	\item Resultados de los tests.
\end{itemize}

Por cada tarea definida, hay que rellenar la hoja de descripción con \textbf{responsable}, \textbf{esfuerzo}, \textbf{duración}, \textbf{recursos}, \textbf{descripción}...

Errores típicos:
\begin{itemize}
	\item Añadir requisitos del producto final.
	\item No se realizan tareas una vez alcanzado el objetivo final.
	\item Desglose poco específico.
\end{itemize}

\subsection{Ejercicios (examen)*}
En el documento word.

%\begin{proposition}
%    This is a proposition.
%\end{proposition}

%\begin{principle}
 %   This is a principle.
%\end{principle}

% Maybe I need to add one more part: Examples.
% Set style and colour later.


\newpage

% ------------------------------------------------------------------------------
% Reference and Cited Works
% ------------------------------------------------------------------------------

\bibliographystyle{IEEEtran}
\bibliography{References.bib}

% ------------------------------------------------------------------------------

\end{document}