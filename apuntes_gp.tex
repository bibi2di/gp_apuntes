\documentclass{article}

\usepackage{amsmath, amsthm, amssymb, amsfonts}
\usepackage{thmtools}
\usepackage{graphicx}
\usepackage{setspace}
\usepackage{geometry}
\usepackage{float}
\usepackage{hyperref}
\usepackage[utf8]{inputenc}
\usepackage[english]{babel}
\usepackage{framed}
\usepackage[dvipsnames]{xcolor}
\usepackage{tcolorbox}

\colorlet{LightGray}{White!90!Periwinkle}
\colorlet{LightOrange}{Orange!15}
\colorlet{LightGreen}{Green!15}

\newcommand{\HRule}[1]{\rule{\linewidth}{#1}}

\declaretheoremstyle[name=Definicion,]{thmsty}
\declaretheorem[style=thmsty,numberwithin=section]{theorem}
\tcolorboxenvironment{theorem}{colback=LightGray}

\declaretheoremstyle[name=Proposition,]{prosty}
\declaretheorem[style=prosty,numberlike=theorem]{proposition}
\tcolorboxenvironment{proposition}{colback=LightOrange}

\declaretheoremstyle[name=Principle,]{prcpsty}
\declaretheorem[style=prcpsty,numberlike=theorem]{principle}
\tcolorboxenvironment{principle}{colback=LightGreen}

\setstretch{1.2}
\geometry{
    textheight=9in,
    textwidth=5.5in,
    top=1in,
    headheight=12pt,
    headsep=25pt,
    footskip=30pt
}

% ------------------------------------------------------------------------------

\begin{document}

% ------------------------------------------------------------------------------
% Cover Page and ToC
% ------------------------------------------------------------------------------

\title{ \normalsize \textsc{}
		\\ [2.0cm]
		\HRule{1.5pt} \\
		\LARGE \textbf{\uppercase{Apuntes Gestión de Proyectos}
		\HRule{2.0pt} \\ [0.6cm] \LARGE{} \vspace*{10\baselineskip}}
		}
\date{}
\author{\textbf{Author} \\ 
		B.L.B}

\maketitle
\newpage

\tableofcontents
\newpage

% ------------------------------------------------------------------------------
\section{Conceptos básicos}
\subsection{Proyecto}
\begin{theorem}
Esfuerzo temporal que se lleva a cabo para crear un producto, servicio o resultado único.
\end{theorem}
Características:
\begin{description}
\item[Temporal]
	\begin{itemize}
		\item Siempre tiene un inicio y un final.
		\item Inicio: se decide \textbf{qué} hacer.
		\item Final: se obtiene el producto/servicio.
		\item Algunos se cancelan
	\end{itemize}
\item[Crean un producto único y medible]
\item[Progresivo]
\end{description}
Los proyectos \textbf{no} son siempre críticos o estratégicos, \textbf{no} son procesos y \textbf{no} son siempre exitosos.

\subsection{Dirección de Proyectos}
\begin{theorem}
Aplicación de conocimientos, habilidades, herramientas y técnicas a las actividades de un proyecto para satisfacer sus requisitos.
\end{theorem}
El \textit{director del proyecto} es la persona responsable de alcanzar los objetivos del proyecto. Para ello necesita:
\begin{itemize}
	\item Conocimientos.
	\item Capacidad para sacar adelante el trabajo.
	\item Habilidades interpersonales.
\end{itemize}

\subsection{Interesados (Stakeholders)}
\begin{theorem}
Personas y organizaciones implicadas de forma activa en dicho proyecto, y aquellos cuyos intereses pueden verse afectados de manera positiva o negativa como resultado del desarrollo del proyecto. 
\end{theorem}
Por ejemplo: equipo de proyecto, empleador, usuarios, sociedad, etc.

\subsection{Conceptos básicos}
\begin{theorem}
\textbf{Requisitos:} Necesidades y expectativas de los interesados.
\end{theorem}

\begin{theorem}
\textbf{Objetivo: } Qué queremos obtener como consecuencia del desarrollo del proyecto.
\end{theorem}

\begin{theorem}
\textbf{Alcance: }Acciones llevadas a cabo para conseguir los objetivos.
\end{theorem}

\begin{theorem}
\textbf{Recursos: } Elementos necesarios para el desarrollo del proyecto. Estos recursos son limitados y hay que gestionarlos de manera adecuada.
\end{theorem}

\begin{theorem}
\textbf{Calidad: }Conjunto de características que deseamos que cumplan los requisitos de los productos.
\end{theorem}

\begin{theorem}
\textbf{Riesgo: }Todo aquello que puede ir mal en la consecución de los objetivos.
\end{theorem}

\subsection{Restricciones del proyecto}
En todo proyecto se sufrirá el efecto de seis tipos de restricciones: tiempo, coste, alcance, recursos, calidad y riesgo. Cada vez que ocurra un cambio en el proyecto, habrá que mirar cuál es el efecto que tiene sobre cada uno de los tipos de restricciones.

\subsection{Proceso:}
\begin{theorem}
Serie de acciones que dan lugar a un resultado.
\end{theorem}
Un proyecto se compone de procesos relacionados entre sí mediante sus entradas y salidas. Toda tarea de un proyecto pertenece a un proceso.
Hay dos categorías principales:
\subsubsection{Procesos orientados a producto}
Contemplan la especificación, diseño y creación del objetivo del proyecto y varían según el área de aplicación y el objetivo. Del objetivo se decide un ciclo de vida y este especifica los procesos a ejecutar.
\subsubsection{Procesos de Planificación y Gestión}
Tratan de la planificación, organización y gestión del proyecto. La mayoría de estos procesos son aplicables a todos los proyectos. La guía del PMBOK define 47 procesos y los clasifica de la siguiente manera:
\begin{itemize}
	\item 5 grupos de procesos (en base a su funcionalidad).
	\begin{itemize}
		\item Procesos de inicio: Se utilizan para \textbf{definir} un nuevo proyecto. Se especifican decisiones de alto nivel (escoger director, justificación del proyecto, restricciones, supuestos...). Se identifican a los \textbf{interesados}
		\item Procesos de planificación: tienen como objetivo obtener el plan de proyecto (identificar tareas, estimación de esfuerzo, planificar recursos...).
		\begin{itemize}
		\item \textbf{Plan de proyecto:} documento coherente y firme que sirve de guía para la ejecución y control del proyecto.
		\item \textbf{DOP:} se desarrolla en las primeras fases del proyecto, contiene menos información que el plan de proyecto y debe ser aprobado formalemente. Incluye descripción, objetivos, alcance, planificación temporal, riestos y método de trabajo.
		\end{itemize}
		\item Procesos de ejecución: desarrollan el trabajo necesario para conseguir los objetivos establecidos siguiendo el plan del proyecto.
		\item Procesos de seguimiento y control: se encargan del seguimiento del trabajo realizado. Se compara el estado de las tareas, recursos utilizados, etc con el plan del proyecto. Cuando sea necesario, se solicitarán modificaciones. 
		\item Procesos de cierre: se verifica que han finalizado los procesos anteriores, se obtiene el visto bueno del cliente...
	\end{itemize}
	\item 10 áreas de conocimiento (en base a su temática).
	\begin{itemize}
		\item Integración
		\item Alcance
		\item Tiempo
		\item Coste
		\item Calidad
		\item Recursos Humanos
		\item Comunicación
		\item Riesgos
		\item Adquisiciones/Suministros
		\item Interesados
	\end{itemize}
\end{itemize}
\section{Reuniones}
En la primera reunión interna se hará partícipe al equipo de los objetivos de proyecto, se asignarán responsabilidades, se organizará la gestión del proyecto y se preparará la reunión con el cliente.
\subsection{Reuniones con cliente}
Se revisarán objetivos, se concretará qué hacer y qué no hacer y se concretarán los requisitos. 
Se intentará conocer al cliente antes de la reunión, y cuando se trate de aspectos técnicos, debemos intentar dirigir la reunión nosotros.

\subsection{Reuniones de desarrollo}
Sirven para especificar el trabajo a desarrollar y distribuir responsabilidades. Los posibles asistentes son: los miembros del equipo, el cliente, terceros implicados...

\subsection{Reuniones de seguimiento}
Son sólo para miembros del equipo. Se dispondrá de la programación del proyecto y de la asignación de recursos en cada instante. Se conocerán los costes hasta el momento. 
Es necesario realizar un seguimiento de lo planificado, tomando las medidas oportunas cuando:
\begin{itemize}
	\item Se produzcan retrasos.
	\item Costes por encima de lo planificado.
	\item Se contravenga algunas condiciones acordadas que fueron base en la decisión de realizar este proyecto. 
\end{itemize}
Tan pronto como se observen desviaciones habrá que replanificar o renegociar el plan del proyecto con los clientes. 
\subsection{Roles}
\begin{itemize}
	\item Coordinador: planifica la reunión, define la agenda, el orden del día, controla el desarrollo de la reunión y la evalúa. 
	\item Secretario: convoca la reunión, prepara los materiales y redacta el acta de reunión.
	\item Asistentes
\end{itemize}
\subsection{Antes de la reunión}
Planificar la reunión:
\begin{itemize}
	\item Objetivo: ¿Para qué hacemos la reunión?
	\item Agenda: ¿Qué temas vamos a tratar? ¿Qué queremos transmitir?
	\item Asistentes: ¿A quién va dirigida la reunión? ¿Quién pondrá pegas durante la reunión?
	\item Materiales: ¿Qué documentación hay que preparar?
	\item Organización de la agenda: orden de cada tema, tiempo que dedicaremos a cada uno.
	\item Lugar, fecha, hora, duración.
	\item Preparar los materiales (actas, documentación...)
	\item Convocatoria formal de la reunión (fecha, hora, convocados, orden del día...)
\end{itemize}
\subsection{Durante la reunión}
\begin{itemize}
	\item Exponer los objetivos
	\item Aprobar las actas pendientes
	\item Tratar los temas de la agenda
	\item Mantener la buena dirección de la reunión
	\item Concluir adecuadamente la reunión (resumir, planificar...)
\end{itemize}
\subsection{Después de la reunión}
\begin{itemize}
	\item Elaborar el acta
	\item Archivar el material generado
	\item Enviar el material a quien corresponda
	\item Anunciar la próxima reunión
	\item Preparar la siguiente reunión
\end{itemize}
\subsection{DO's Convocatorias}
\begin{itemize}
	\item Asegurarse que la reunión no se puede sustituir con un mail, llamada, etc.
	\item Convocar un día y una hora con alternativa.
	\item Identificar todos los asistentes necesarios.
	\item Proponer una agenda concreta con puntos relevantes.
	\item Identificar los temas de interés.
\end{itemize}
\subsection{DONT's Convocatorias}
\begin{itemize}
	\item Fijar reuniones sin causa real.
	\item Preguntar a cada asistente intentando fijar una fecha.
	\item Invitar prescindibles / Olvidar imprescindibles
	\item Proponer agenda vaga.
\end{itemize}
\subsection{DO's Actas}
\begin{itemize}
	\item Marcar fechar, hora, lugar y asistentes.
	\item Apuntar datos de interés.
	\item Identificar al autor de cada comentario.
	\item Sintetizar al máximo.
	\item Apuntar las decisiones tomadas.
\end{itemize}
\subsection{DONT's Convocatorias}
\begin{itemize}
	\item Introducir opiniones.
	\item Describir hechos no relacionados.
	\item Tomar partido.
	\item No reflejar las posturas de las distintas partes.
\end{itemize}

\section{Gestión de Recursos}
\begin{theorem}
Asignación de recursos: asociar a cada tarea las personas y materiales necesarios para llevarlas a cabo.
\end{theorem}
\subsection{Recursos humanos}
Los recursos humanos son el componente económico más importante del proyecto. La mayoría de proyectos implican trabajo en equipo y hay que tener especial cuidado en gestionar estos recursos.
Tipos de personalidad en un proyecto:
\begin{itemize}
	\item Orientada a tarea: mejores técnicamente.
	\item Orientada a la relación: falcilitan la comunicación.
	\item Orientada a sí mismo: mejores para realizar la tarea a tiempo.
\end{itemize}
El equipo más efectivo sería uno equilibrado liderado por alguien orientado a tarea.
Distribución del tiempo:
\begin{itemize}
	\item Actividades no productivas 20%
	\item Trabajo individual 30%
	\item Interacción con otras personas 50%
\end{itemize}
Existen 3 formas de estructurar la comunicación:
\begin{itemize}
	\item Individual: cada persona desarrolla una tarea y sólo responde ante el jefe del proyecto. El jefe del proyecto coordina las tareas. N enlaces de comunicación.
	\item Democrático: las tareas se asignan a los equipos donde los miembros cooperan para desarrollarlas. La organización, coordinación y distribución se realizan conjuntamente y un miembro del equipo será el portavoz ante el jefe. 1+N*(N-1)/2 enlaces.
	\item Jerárquico: existen jefes de grupo que coordinan y asignan tarea a los miembros. Los jefes de grupo rinden cuentas al jefe de proyecto. 
\end{itemize}
La organización jerárquica es la más eficiente. En casos de proyectos muy pequeños o con partes muy independientes no es práctico usar equipos. \\
\subsubsection{Esfuerzo vs. Duración}
\begin{theorem}
Esfuerzo: tiempo que una persona necesita para realizar una tarea (persona-hora, persona-día, persona-mes)
\end{theorem}
\begin{theorem}
Duración: Tiempo que transcurre desde que comienza una tarea hasta que finaliza. 
\end{theorem}
Una tarea requiere 12 personas-hora. Tenemos 3 personas. Duració mínima: 4 horas.\\
La relación entre esfuerzo/duración no siempre es proporcional al número de personas. Existen diferentes situaciones y las comunicaciones no siempre son las mismas. 
\begin{itemize}
	\item \textbf{Subtareas independientes:} no hay necesidad de comunicación, durará menos cuantas más personas haya.
	\item \textbf{Tarea indivisible: } duración se mantiene constante.
	\item \textbf{Tarea divisible con comunicación: } la duración disminuye con el número de personas.
	\item \textbf{Tarea divisible con interrelaciones complejas: }la duración puede aumentar con las personas.
\end{itemize}
En general, es mejor disponer de un equipo pequeño de buenos profesionales. Con las personas correctas, aún con herramientas, lenguajes y procesos insuficientes, se puede tener éxito. Hay que equilibrar al personal. 
\begin{itemize}
	\item \textbf{KSA, capacidad técnica: } incluye experiencia sobre la materia, conocimientos para realizar la tarea y la capacidad de realizarla. 
	\item \textbf{MAC, voluntad: }incluye la motivación personal, el compromiso que asumirá y la seguridad en sí mismo para realizarla. 
\end{itemize}
Posibles situaciones sobre realizar el trabajo: 
\begin{itemize}
	\item Puede y quiere - ideal
	\item Puede y accede - pensar en tareas de motivación
	\item Puede y no está dispuesto - resolver el problema
	\item Puede ser formado - supone gastos, modificaciones, riesgos
	\item No puede - adjudicarle otra tarea
\end{itemize}
Qué hay que hacer:
\begin{itemize}
	\item Asegurar que la capacidad de desarrollo no se centre en unos pocos individuos. 
	\item Equilibrar las motivaciones individuales y las organizacionales.
	\item Retener personal con conocimientos y habilidades críticas. 
\end{itemize}

\subsection{Recursos Materiales}
Es necesario proporcionar materiales adecuados para el desarrollo de tareas. El material de trabajo tiene que estar en buenas condiciones y debe haber políticas de renovación de materiales. \\
\textbf{El lugar de trabajo es un recurso material.}\\
Diferentes aspectos que afectan a la productividad:
\begin{itemize}
	\item Intimidad: para concentrarse y trabajar sin interrupciones.
	\item Conciencia exterior: luz natural y vista al exterior. 
	\item Personalización: posibilidad de adaptar al gusto propio el lugar de trabajo personal.
	\item Comodidad: el trabajador tiene que estar cómodo. 
\end{itemize}
\section{Objetivos y alcance (06-02-2024)}

Son elementos clave para definir y planificar el trabajo:
\begin{itemize}
	\item Objetivos (QUÉ): resultados esperados. ¿Qué queremos conseguir?
	\item Alcance (CÓMO): ámbito de acción y requisitos que debe cumplir. ¿Qué hay que hacer para desarrollar el proyecto?
\end{itemize}

Deben estar alineados entre sí y con la misión del proyecto, y coherentes con los recursos y restricciones existentes. 

\subsection{Definir objetivos}
Pasos:
\begin{itemize}
	\item Identificar interesados: stakeholders
	\item Identificar qué quiere cada interesado
	\item Tener en cuenta solo los intereses convenientes para el proyecto. 
\end{itemize}
\subsection{Redactar objetivos}
La redacción de los objetivos es única para todos los interesados.
\begin{itemize}
	\item Objetivos SMART: \textbf{específicos}, \textbf{medibles}, \textbf{alcanzables}, \textbf{relevantes}, \textbf{temporales} (se utilizan verbos para redactarlos)
	\item Objetivos \textbf{generales} y \textbf{específicos}:
	\begin{itemize}
		\item Generales: declaraciones claras y concisas de los propósitos del proyecto.
		\item Específicos: se definen las metas a corto plazo que deben cumplirse para lograr el objetivo general de forma más detallada.
	\end{itemize}
	\item Errores típicos:
	\begin{itemize}
		\item Dividir los objetivos por interesados.
		\item Confundir tareas con objetivos.
		\item Definir objetivos poco realistas o inalcanzables.
	\end{itemize}
\end{itemize}

\subsection{Alcance}

Una vez definidos los objetivos, hay que definir qué pasos dar para cumplirlos.
\begin{itemize}
	\item El alcance debe ser claro, completo y consensuado con las partes interesadas
	\item Alcance del producto: características y funcionalidades del producto final: Documento de requisitos.
	\item Alcance del proyecto: tareas que se llevarán a cabo para conseguir el producto: Estructura de Descomposición del Trabajo (EDT).
\end{itemize}

\subsection{Redactar alcance}
\begin{itemize}
	\item Ver qué tareas son necesarias para conseguir los objetivos
	\item Dividir las tareas en subtareas más fáciles de estimar y asignar al personal.
	\item Las tareas se incluyen en una estructura jerárquica llamada \textbf{EDT}.
\end{itemize}

\subsection{EDT}
\begin{itemize}
	\item Nivel 1: Proyecto
	\item Nivel 2: Fases, Paquetes de trabajo: INICIO, DISEÑO, IMPLEMENTACIÓN, PRUEBAS, GESTIÓN
	\item Nivel 3: Paquetes de trabajo, tareas que cuelgan del nivel 2.
	\item Nivel 4: Paquetes de trabajo, subtareas que cuelgan del nivel 3.
\end{itemize}
Todas las tareas del EDT forman el alcance del proyecto, y lo que no está en el EDT no es parte del proyecto (todo debe estar definido aquí). 
Gestionar el proyecto (el seguimiento y supervisión) debe estar incluído, al fin y al cabo es tiempo invertido en el proyecto. 

Se pueden establecer una serie de entregas intermedias (entregables): para el cliente (presupuestos, prototipos...) o para el equipo de trabajo (DOP, análisis...) que ayudan a establecer hitos o fechas clave. 

Entregables:
\begin{itemize}
	\item DOP: una vez finalizada la tarea ``Elaboración del DOP".
	\item Presupuesto: una vez finalizada la tarea ``Organización".
	\item Resultados de los tests.
\end{itemize}

Por cada tarea definida, hay que rellenar la hoja de descripción con \textbf{responsable}, \textbf{esfuerzo}, \textbf{duración}, \textbf{recursos}, \textbf{descripción}...

Errores típicos:
\begin{itemize}
	\item Añadir requisitos del producto final.
	\item No se realizan tareas una vez alcanzado el objetivo final.
	\item Desglose poco específico.
\end{itemize}

\subsection{Ejercicios (examen)*}
En el documento word.
\section{Estimación de esfuerzo (13-02-2024)}
Queremos conocer el esfuerzo que supone desarrollar un sistema. 
Métodos para la estimación:

\subsection{Experiencia}
\begin{itemize}
	\item \textbf{Juicio experto - Puro:} Un experto estudia la especificación y hace su estimación. Se basa en sus conocimientos.
	\item \textbf{Juicio experto - Delphi:} Se dan las especificaciones a un grupo de expertos. Cada uno hace una estimación y se remite al coordinador. Este las revisa y si divergen mucho se vuelve a enviar al grupo de expertos. Se repite hasta que converge.
	\item \textbf{Jucio experto - Wideband Delphi:} Mismo que delphi pero se toman la más optimista y la pesimista y se explican.
	\item \textbf{Estimación multipunto:} Se calcula a partir de una media ponderada de estimaciones: optimista, media, pesimista. 
	\item \textbf{Analogía:} Consiste en comparar las especificaciones del proyecto con la de anteriores. Se toman en cuenta diversos factores: tamaño, complejidad, usuarios, SO, HW, entorno... 
	\item \textbf{Distribución de utilización de recursos en el ciclo de vida:} Se revisan las fases, y si son parecidas a las de otro proyecto, se espera que la distribución sea similar. 
\end{itemize}

\subsection{Recursos}
\textbf{Parkinson: }Se ve cuánto personal y de cuánto tiempo se dispone. Después el trabajo se expande hasta consumir todos los recursos disponibles. 

\subsection{Mercado}
\textbf{Precio para vender:} Lo importante es conseguir el contrato, así que el precio se fija en función de lo que el cliente está dispuesto a pagar. Si se usa en conjunción con otros métodos, puede ser aceptable, pero es peligroso como único método de estimación. 

\subsection{Componentes}
\begin{itemize}
	\item \textbf{Bottom-Up: }Se descompone el proyecto en unidades pequeñas.
	\item \textbf{Top-Down: }Se estima el proyecto completo.
\end{itemize}

\subsection{Algoritmos}
\begin{itemize}
	\item \textbf{Puntos de Función:} Se aplica en las primeras fases de desarrollo. Se buscan unas características: entradas, salidas, ficheros y factores de complejidad. 
	\begin{enumerate}
	\item Identificación de \textbf{elementos} y clasificación: FLI, FIE (datos), EE, SE, CE (transacciones)
	\begin{itemize}
		\item \textbf{Ficheros lógicos internos (FLI):} Bases de datos, ficheros
		\item \textbf{Ficheros de Interfaz Externos (FIE):} FLI alojado en otro sistema.
		\item \textbf{Entradas Externas (EE):} datos que llegan desde el exterior al interior, y siempre actualiza un FLI.
		\item \textbf{Salidas Externas (SE):} datos del interior al exterior. Se procesan esos datos, no solo los muestra.
		\item \textbf{Consulta Externa (CE):} proceso formado por una entrada y una salida que no genera datos ni modifica ningún FLI. 
	\end{itemize}
	\item Cálculo de puntos de función no ajustados: se estudia cada elemento de la función, se identifican sus componentes y se calculan los PFNA según las tablas. Casos especiales: pulsar botón, checkbox, radiobutton, combobox, mensaje de error...
	\item Cálculo del factor de ajuste (FA): se calcula basado en 14 características cualitativas generales del software. A cada característica se le asigna un valor del 0-5. FA = 0,65 + (0,01 * SVA) siendo SVA la suma de los valores. 
	\item Cáculo de puntos de función ajustados: PFA = PFNA * FA
	\end{enumerate}
	Después de estos cálculos, se calcula el esfuerzo teniendo en cuenta el lenguaje a programar (y otros factores). Esfuerzo = PFA * Lenguaje (personas/hora).
	\item \textbf{COCOMO II:}  Constructive Cost Model. Es el model de estimación de costes más utilizado. Su objetivo es desarrollar un modelo de estimación de tiempo y de coste del sofwtare de acuerdo con los ciclos de vida más usados, y construir una BD de proyectos de SW que permita la calibración continua del model y así incrementar la precisión de la estimación. 
	Está compuesto por 3 modelos:
	\begin{enumerate}
		\item \textbf{ACM:} Indicado para la etapa de planificación. Se utilizan puntos de objeto, que estiman cuántas pantallas, informes y componentes 3GL que contendrá la aplicación y clasifican cada instancia de un objeto según su nivel de complejidad. Se determinará la cantidad de PO sumando todos los pesos de las instancias. Se estimará el porcentaje de código a reusar. Finalmente, se calcula el ratio de productividad según una tabla. 
		\item \textbf{EDM:} Primeras etapas del proyecto. Cuando se conoce muy poco sobre el tamaño, plataforma y personal del proyecto. Basado en PFNA, que se convierten a miles de líneas de código.
		\item \textbf{PAM:} Modelo más detallado. Tiene 17 factores de ajuste y se usa cuando se ha desarrollado la arquitectura del proyecto. 
	\end{enumerate}
\end{itemize}

\section{Evaluación económica}
\subsection{Estudio financiero}
Tenemos en cuenta la rentabilidad y el coste del proyecto. 
\begin{itemize}
	\item Plazo de recuperación simple
	\item Rendimiento de inversión (ROI):
	\begin{math}
	\sum_{j=1}^{n} \frac{Qj - K}{n * K}*100
	\end{math}
\end{itemize}

\subsection{El valor del dinero}
Para actualizar el valor del dinero se usan estas fórmulas:\\
\begin{math}
Futuro = Actual * (1+i)^n\\
Actual = Futuro * (1+i)^{-n}
\end{math}

Ejemplo: IPC (incremento precios de consumo)

TIR: \begin{math} K=\sum_{j=1}^{n} \frac{Q_j}{(1+r)^j}\end{math}
\\
Son viables los proyectos con r $>$ tipo de interés. Cuanto mayor la r, mejor la inversión.\\
El VAN representa la ganancia neta. Si el VAN es mayor que cero, el proyecto es viable.
\begin{math}
VAN = -K + \sum_{j=1}^{n} \frac{Q_j}{(1+r)^j}
\end{math}

%\begin{proposition}
%    This is a proposition.
%\end{proposition}

%\begin{principle}
 %   This is a principle.
%\end{principle}

% Maybe I need to add one more part: Examples.
% Set style and colour later.



\newpage

% ------------------------------------------------------------------------------
% Reference and Cited Works
% ------------------------------------------------------------------------------

\bibliographystyle{IEEEtran}
\bibliography{References.bib}

% ------------------------------------------------------------------------------

\end{document}