\documentclass{article}

\usepackage{amsmath, amsthm, amssymb, amsfonts}
\usepackage{thmtools}
\usepackage{graphicx}
\usepackage{setspace}
\usepackage{geometry}
\usepackage{float}
\usepackage{hyperref}
\usepackage[utf8]{inputenc}
\usepackage[english]{babel}
\usepackage{framed}
\usepackage[dvipsnames]{xcolor}
\usepackage{tcolorbox}

\colorlet{LightGray}{White!90!Periwinkle}
\colorlet{LightOrange}{Orange!15}
\colorlet{LightGreen}{Green!15}

\newcommand{\HRule}[1]{\rule{\linewidth}{#1}}

\declaretheoremstyle[name=Definicion,]{thmsty}
\declaretheorem[style=thmsty,numberwithin=section]{theorem}
\tcolorboxenvironment{theorem}{colback=LightGray}

\declaretheoremstyle[name=Proposition,]{prosty}
\declaretheorem[style=prosty,numberlike=theorem]{proposition}
\tcolorboxenvironment{proposition}{colback=LightOrange}

\declaretheoremstyle[name=Principle,]{prcpsty}
\declaretheorem[style=prcpsty,numberlike=theorem]{principle}
\tcolorboxenvironment{principle}{colback=LightGreen}

\setstretch{1.2}
\geometry{
    textheight=9in,
    textwidth=5.5in,
    top=1in,
    headheight=12pt,
    headsep=25pt,
    footskip=30pt
}

% ------------------------------------------------------------------------------

\begin{document}

% ------------------------------------------------------------------------------
% Cover Page and ToC
% ------------------------------------------------------------------------------

\title{ \normalsize \textsc{}
		\\ [2.0cm]
		\HRule{1.5pt} \\
		\LARGE \textbf{\uppercase{Apuntes Gestión de Proyectos}
		\HRule{2.0pt} \\ [0.6cm] \LARGE{} \vspace*{10\baselineskip}}
		}
\date{}
\author{\textbf{Author} \\ 
		B.L.B}

\maketitle
\newpage

\tableofcontents
\newpage

% ------------------------------------------------------------------------------

\section{Objetivos y alcance (06/02/2024)}

Son elementos clave para definir y planificar el trabajo:
\begin{itemize}
	\item Objetivos (QUÉ): resultados esperados. ¿Qué queremos conseguir?
	\item Alcance (CÓMO): ámbito de acción y requisitos que debe cumplir. ¿Qué hay que hacer para desarrollar el proyecto?
\end{itemize}

Deben estar alineados entre sí y con la misión del proyecto, y coherentes con los recursos y restricciones existentes. 

\subsubsection{Definir objetivos}
Pasos:
\begin{itemize}
	\item Identificar interesados: stakeholders
	\item Identificar qué quiere cada interesado
	\item Tener en cuenta solo los intereses convenientes para el proyecto. 
\end{itemize}
\subsubsection{Redactar objetivos}
La redacción de los objetivos es única para todos los interesados.
\begin{itemize}
	\item Objetivos SMART: \textbf{específifos}, \textbf{medibles}, \textbf{alcanzables}, \textbf{relevantes}, \textbf{temporales} (se utilizan verbos para redactarlos)
	\item Objetivos \textbf{generales} y \textbf{específicos}:
	\begin{itemize}
		\item Generales: declaraciones claras y concisas de los propósitos del proyecto.
		\item Específicos: se definen las metas a corto plazo que deben cumplirse para lograr el objetivo general de forma más detallada.
	\end{itemize}
	\item Errores típicos:
	\begin{itemize}
		\item Dividir los objetivos por interesados.
		\item Confundir tareas con objetivos.
		\item Definir objetivos poco realistas o inalcanzables.
	\end{itemize}
\end{itemize}

\subsubsection{Alcance}

Una vez definidos los objetivos, hay que definir qué pasos dar para cumplirlos.
\begin{itemize}
	\item El alcance debe ser claro, completo y consensuado con las partes interesadas
	\item Alcance del producto: características y funcionalidades del producto final: Documento de requisitos.
	\item Alcance del proyecto: tareas que se llevarán a cabo para conseguir el producto: Estructura de Descomposición del Trabajo (EDT).
\end{itemize}

\subsubsection{Redactar alcance}
\begin{itemize}
	\item Ver qué tareas son necesarias para conseguir los objetivos
	\item Dividir las tareas en subtareas más fáciles de estimar y asignar al personal.
	\item Las tareas se incluyen en una estructura jerárquica llamada \textbf{EDT}.
\end{itemize}

\subsubsection{EDT}
\begin{itemize}
	\item Nivel 1: Proyecto
	\item Nivel 2: Fases, Paquetes de trabajo: INICIO, DISEÑO, IMPLEMENTACIÓN, PRUEBAS, GESTIÓN
	\item Nivel 3: Paquetes de trabajo, tareas que cuelgan del nivel 2.
	\item Nivel 4: Paquetes de trabajo, subtareas que cuelgan del nivel 3.
\end{itemize}
Todas las tareas del EDT forman el alcance del proyecto, y lo que no está en el EDT no es parte del proyecto (todo debe estar definido aquí). 
Gestionar el proyecto (el seguimiento y supervisión) debe estar incluído, al fin y al cabo es tiempo invertido en el proyecto. 

Se pueden establecer una serie de entregas intermedias (entregables): para el cliente (presupuestos, prototipos...) o para el equipo de trabajo (DOP, análisis...) que ayudan a establecer hitos o fechas clave. 

Entregables:
\begin{itemize}
	\item DOP: una vez finalizada la tarea "Elaboración del DOP".
	\item Presupuesto: una vez finalizada la tarea "Organización".
	\item Resultados de los tests.
\end{itemize}

Por cada tarea definida, hay que rellenar la hoja de descripción con \textbf{responsable}, \textbf{esfuerzo}, \textbf{duración}, \textbf{recursos}, \textbf{descripción}...

Errores típicos:
\begin{itemize}
	\item Añadir requisitos del producto final.
	\item No se realizan tareas una vez alcanzado el objetivo final.
	\item Desglose poco específico.
\end{itemize}

\subsubsection{Ejercicios (examen)*}
En el documento word.

%\begin{proposition}
%    This is a proposition.
%\end{proposition}

%\begin{principle}
 %   This is a principle.
%\end{principle}

% Maybe I need to add one more part: Examples.
% Set style and colour later.


\newpage

% ------------------------------------------------------------------------------
% Reference and Cited Works
% ------------------------------------------------------------------------------

\bibliographystyle{IEEEtran}
\bibliography{References.bib}

% ------------------------------------------------------------------------------

\end{document}